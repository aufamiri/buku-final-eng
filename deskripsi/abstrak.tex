% Mengubah keterangan `Abstract` ke bahasa indonesia.
% Hapus bagian ini untuk mengembalikan ke format awal.
\renewcommand\abstractname{Abstrak}

\begin{abstract}

  % Ubah paragraf berikut sesuai dengan abstrak dari penelitian.
  Fake news or called hoax, is one of the things that still plaguing Indonesia. Even more so, with the rise of the social media, a fake news can spread wider and faster than ever before. Worse, Indonesian people have quite a high tendencies to share fake news. That is why, we are in a dire need of a method to detect fake news. This research is using BERT to automatically classifiy whether a news is a hoax or not. From a raw text, we applied a tokenization process before inputting the text to the BERT. Next, the pooled output of the BERT is being used as the input for Linear Regression, a tested-and-true method for classifying task. The output of the Linear Regression is then being used as a way to determine whether a news is a hoax or not. The purpose of this research is to create a machine learning model to help the people to determine whether a text is a fake news or not. The result of this research is a model to classify a hoax text with 89\% in accuracy.

\end{abstract}

% Mengubah keterangan `Index terms` ke bahasa indonesia.
% Hapus bagian ini untuk mengembalikan ke format awal.
\renewcommand\IEEEkeywordsname{Kata kunci}

\begin{IEEEkeywords}

  % Ubah kata-kata berikut sesuai dengan kata kunci dari penelitian.
  BERT, Hoax, Fake News Classification, Linear Regression

\end{IEEEkeywords}
