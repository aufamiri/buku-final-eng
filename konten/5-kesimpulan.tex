% Ubah judul dan label berikut sesuai dengan yang diinginkan.
\section{Conclusion}
\label{sec:conclusion}

% Ubah paragraf-paragraf pada bagian ini sesuai dengan yang diinginkan.
From the entire experiment, there are a few things in which it can be concluded with has been listed below :

\begin{enumerate}
    \item The more dataset is being used on its pre-training phase of BERT, the more accurate a model is. This has been proved by model \textit{indobert-base-p1} in which the model has been pre-trained with more than 23GB worth of data.

    \item Best truncating method in our approach is by truncating only the first few sentences of an entire news. This approach has successfully obtained the highest accuracy with the the gap of 3\% on nearly all metrics if compared to other truncation approach. This is most likely because more often than not, the Indonesian news site is started with a lead or a shorter and denser news content written in a single paragraph, and by taking the starting part of the news text, the lead has been included into the processed text as well.

    \item
\end{enumerate}

\lipsum[21-23]
